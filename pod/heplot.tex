
% translated by spod2latex from heplot.pod on Mon Nov 13 09:30:54 EST 2006
% translated by spod2latex from heplot.pod on Fri Sep 29 08:44:34 EDT 2006
\subsection{The \macro{HEPLOT}: Plot hypothesis and error matrices for a bivariate MANOVA effect }\label{mac:heplot}
 \index{\macro{heplot}}

The \macro{HEPLOT} plots the covariance ellipses for a hypothesized (H)
effect and for error (E) for two variables from a MANOVA.  The plot helps
to show how the means of the groups differ on the two variables
jointly, in relation to the within-group variation.  The test statistics
for any MANOVA are essentially saying how 'large' the variation in
H is, relative to the variation in E, and in how many dimensions.
The HEPLOT shows a two-dimensional visualization of the answer to this
question.  An alternative two-dimensional view is provided by the
\macro{CANPLOT}, which shows the data, variables, and within-group ellipses
projected into the space of the largest two canonical variables---
linear combinations of the responses for which the group differences
are largest.

Typically, you perform a MANOVA analysis with \PROC{GLM}, and save the
output statistics, including the H and E matrices, using the \texttt{OUTSTAT=}
option.  This must be supplied to the macro as the value of the
\texttt{STAT=} parameter.  If you also supply the raw data for the analysis
via the \texttt{DATA=} parameter, the means for the levels of the \texttt{EFFECT=}
parameter are also shown on the plot.

Various kinds of plots are possible, determined by the \texttt{M1=} and
\texttt{M2=} parameters.  The default is \texttt{M1=H} and \texttt{M2=E}.
If you specify \texttt{M2=I} (identity matrix), then the \mat{H}
and \mat{E} matrices are transformed to $\mat{H}^* = \mat{e}\mat{H}\mat{e}$ (where
$\mat{e}=\mat{E}^{-1/2}$), and $\mat{E}^*=\mat{e}\mat{E}\mat{e}=\mat{I}$,
so the errors become uncorrelated, and the size of $\mat{H}^*$ can be
judged more simply in relation to a circular $\mat{E}^*=\mat{I}$. For
multi-factor designs, is it sometimes useful to specify \texttt{M1=H+E},
so that each factor can be examined in relation to the within-cell
variation.

\subsubsection*{Usage}

The \macro{HEPLOT} is defined with keyword parameters.  The \texttt{STATS=}
parameter and either the \texttt{VAR=} or the \texttt{X=} and \texttt{Y=} parameters are required.
You must also specify the \texttt{EFFECT=} parameter, indicating the H matrix
to be extracted from the \texttt{STATS=} data set.
The arguments may be listed within parentheses in any order, separated
by commas. For example:

\begin{listing}
 proc glm data=dataset outstat=HEstats;
    model y1 y2  = A B A*B / ss3;
    manova;
 %heplot(data=dataset, stat=HEstats, var=y1 y2, effect=A );
 %heplot(data=dataset, stat=HEstats, var=y1 y2, effect=A*B );
\end{listing}

\subsubsection*{Parameters}
\begin{proglist}

\item[STAT=] Name of the \texttt{OUTSTAT=} dataset from \PROC{GLM} containing the
SSCP matrices for model effects and ERROR, as indicated by
the \verb|_SOURCE_| variable.

\item[DATA=] Name of the input, raw data dataset (for means)

\item[X=] Name of horizontal variable for the plot

\item[Y=] Name of vertical variable for the plot

\item[VAR=] 2 response variable names: x y.  Instead of specifying \texttt{X=}
and \texttt{Y=} separately, you can specify the names of two response
variables with the \texttt{VAR=} parameter.

\item[EFFECT=] Name of the MODEL effect to be displayed for the H matrix.
This must be one of
the terms on the right hand side of the MODEL statement used
in the \PROC{GLM} or \PROC{REG} step, in the same format that this
efffect is labeled in the \texttt{STAT=} dataset. This must be one of
the values of the \verb|_SOURCE_| variable contained in the \texttt{STAT=} 
dataset.

\item[CLASS=] Names of class variables(s), used to find the means for groups
to be displayed in the plot.  The default value is the value
specified for EFFECT=, except that '*' characters are changed
to spaces. Set \texttt{CLASS=} (null) for a quantitative regressor or to
suppress plotting the means.

\item[EFFLAB=] Optional label (up to 16 characters) for the H effect, annotated near the upper
corner of the H ellipse

\item[MPLOT=] Matrices to plot. \texttt{MPLOT=1} plots only the H ellipse. \default{MPLOT=1 2}

\item[GPFMT=] The name of a SAS format for levels of the group/effect variable used in labeling
group means.

\item[ALPHA=] Non-coverage proportion for the ellipses. \default{ALPHA=0.32}

\item[PVALUE=] Coverage proportion, 1-alpha. \default{PVALUE=0.68}

\item[SS=] Type of SS to extract from the \texttt{STAT=} dataset. The possibilities
are SS1-SS4, or CONTRAST (but the SSn option on the MODEL statement in
\PROC{GLM} will limit the types of SSCP matrices produced).
This is the value of the \verb|_TYPE_| variable in the \texttt{STAT=} dataset.
\default{SS=SS3}

\item[WHERE=] To subset both the \texttt{STAT=} and \texttt{DATA=} datasets

\item[ANNO=] Name of an input annotate data set, used to add additional
information to the plot of y * x.

\item[ADD=] Specify \texttt{ADD=CANVEC }to add canonical vectors to the plot. The
\PROC{GLM} step must have included the option CANONICAL on the
MANOVA statement.

\item[M1=] First matrix: either H or H+E. \default{M1=H}

\item[M2=] Second matrix either E or I. \default{M2=E}

\item[SCALE=] Scale factors for M1 and M2.  This can be a pair of numeric
values or expressions using any of the scalar values calculated
in the \PROC{IML} step.  The default scaling [\texttt{SCALE=1 1}]
results in a plot of E/dfe and H/dfe, where the size 
and orientation of E shows error variation on the data scale,
and H is scaled conformably, allowing the group means to be
shown on the same scale. The \emph{natural scaling} of H and E
as generalized mean squares would be H/dfh and E/dfe, which is
obtained using \texttt{SCALE=dfe/dfh 1}, Equivalently, the E matrix can
be shrunk by the same factor by specifying \texttt{SCALE=1 dfh/dfe}.

\item[VAXIS=] Name of an AXIS statement for the y variable

\item[HAXIS=] Name of an AXIS statement for the x variable

\item[LEGEND=] Name of a LEGEND statement.  If not specified, a legend for
the M1 annd M2 matrices is drawn beneath the plot. Specify LEGEND=NONE
to suppress the legend.

\item[COLORS=] Colors for the H and E ellipses. \default{COLORS=BLACK RED}

\item[LINES=] Line styles for the H and E ellipses. \default{LINES=1 21}

\item[WIDTH=] Line widths for the H and E ellipses. \default{WIDTH=3 2}

\item[HTEXT=] Height of text in the plot.  If not specified, the global
graphics option HTEXT controls this.

\item[OUT=] Name of the output dataset containing the points on the
H and E ellipses. \default{OUT=OUT}

\item[NAME=] Name of the graphic catalog entry. \default{NAME=HEPLOT}

\item[GOUT=] Name of the graphic catalog. \default{GOUT=GSEG}

\end{proglist}
