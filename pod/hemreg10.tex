
% translated by spod2latex from hemreg.pod on Fri Sep 29 16:16:39 EDT 2006
\subsection{The \macro{HEMREG}: Extract H and E matrices for multivariate regression}\label{mac:hemreg}
 \index{\macro{hemreg}}

The \macro{HEMREG} extracts hypothesis (H) and error (E) matrices for an
overall test in a multivariate regression analysis, in a form similar to
that provided by the \texttt{OUTSTAT=} option with \PROC{GLM}.  This is typically
used with the \macro{HEPLOT}, or the \macro{MPOWER} for MMRA.

\subsubsection*{Method}

For a multivariate regression analysis, using

\begin{listing}
 proc glm outstat=stats;
    model y1 y2 y3 = x1-x5;
\end{listing}

\PROC{GLM} will produce 5 separate 3x3, 1 df SSCP matrices for the separate
predictors X1-X5, in the \texttt{OUTSTAT=} data set, but no SSCP matrix for
the overall multivariate test. The \macro{HEMREG} uses \PROC{REG} instead,
obtains the HypothesisSSCP and ErrorSSCP tables using ODS, and massages
these into the same format used by \PROC{GLM}.

\subsubsection*{Usage}

The \macro{HEMREG} is defined with keyword parameters.  The \texttt{Y=} and
\texttt{X=} parameters are required.
The arguments may be listed within parentheses in any order, separated
by commas. For example:

\begin{listing}
 %hemreg(y=SAT PPVT RAVEN, x=N S NS NA SS);
\end{listing}
\subsubsection*{Parameters}
\begin{proglist}

\item[DATA=] Name of input dataset. \default{DATA=_LAST_}

\item[Y=] List of response variables.  Must be an explicit,
blank-seaparated list of variable names.

\item[X=] List of predictor variables.  Must be an explicit,
blank-seaparated list of variable names.

\item[HYP=] Hame for the overall hypothesis \verb|_SOURCE_| (to be used as the \texttt{EFFECT=}
parameter in the \macro{HEPLOT}.  \default{HYP=HYP}

\item[OUT=] The name of output HE dataset. \default{OUT=HE}

\end{proglist}
