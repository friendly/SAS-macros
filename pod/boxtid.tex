
% translated by spod2latex from boxtid.pod
\section{The \macro{BOXTID}: Power transformations by Box-Tidwell method}\label{mac:boxtid}

The \macro{BOXTID} finds power transformations for some or all of the
predictors in a regression model using the Box-Tidwell method.  In
addition, it can produce plots showing the influence of individual
observations on the selection of powers. These are partial residual
plots for the constructed variables X * log X.

As a convenience, an output data set containing the optimally
transformed variables is also produced.

\subsection*{Usage}

The \macro{BOXTID} takes 14 keyword arguments.  You must specify either
the \texttt{RESP=} or \texttt{YVAR=} parameter, and the names of all predictors (\texttt{XVAR=}).
For example:

\begin{listing}
  %boxtid(data=angell, yvar=moralint,
       xvar=hetero mobility, id=city);
\end{listing}
\subsubsection*{Parameters}

Default values are shown after the parameter name.

\begin{proglist}

\item[DATA=\_last\_] Name of input data set

\item[RESP=] The name of the response variable

\item[YVAR=] Response variable (synonym for \texttt{RESP=})

\item[XVAR=] Names of the predictors in the model.  This must
be a simple list of variable names, i.e., lists
like \texttt{X1-X10} are not allowed.

\item[XTRANS=] Variables to be transformed: names or indices.
If \texttt{XVAR=X1 X2 X3 X7 X9}, you may specify either
\texttt{XTRANS=X3 X7 X9} or \texttt{XTRANS=3 4 5} for the same effect.
If not specified, all variables in the \texttt{XVAR=} list
are transformed.

\item[PREFIX=t\_] Prefix for names of transformed variables.  If the
X variables are X1 X2 X3, the output data set will
contain T\_X1, T\_X2, T\_X3 when the \texttt{PREFIX=T\_}.

\item[ID=] Name of an ID variable, used as a point label in plots.

\item[OUT=boxtid] Name of output data set

\item[ROUND=0.5] Round powers.  The estimated power for each predictor
is rounded to the nearest \texttt{ROUND=} unit in constructing
the transformed variables.

\item[MAXIT=15] Maximum number of iterations

\item[CONVERGE=0.001] Convergence criterion.  The process stops when the
largest change in an estimated power is less than
the \texttt{CONVERGE=} value, or when \texttt{MAXIT} iterations would
be exceeded.

\item[PPLOT=] Specifies printer plots, if any to be produced.
Either or both of the keywords TRANS and INFL.

\item[GPLOT=INFL] Specifies high-res influence plots.

\item[QUIET=N] Y or N.  \texttt{QUIET=Y} suppresses printout of the iteration
history.

\end{proglist}
